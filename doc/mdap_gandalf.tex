\section{{\tt mdap\_gandalf.pro}}
\label{dap_sec:mdap_gandalf}

This main module is an implementation of the original Gandalf program by
Sarzi et al. (2006). 

Modifications to the original procedure include:

\begin{itemize}

   \item It takes into account different values of the instrumental
     velocity dispersion for each emission lines (i.e. INST\_DISP is
     function of wavelenght).

   \item It is possible to fix the gas kinematics to the starting
     guesses (fix\_gas\_kin keyword), or specify the best fit
     boundaries (range\_v\_gas and range\_s\_gas keywords).

   \item Inclusion of the optional keyword /external\_library, to run
     the fortran version of the BVLS module.

   \item convolution is performed with the same utility module used by
     pPXF: mdap\_ppxf\_convol\_fft.pro

\end{itemize}

Table \ref{mdap_tab:mdap_gandalf} lists the inputs and outputs of the
{\tt mdap\_gandalf.pro} module.

\begin{center}
\begin{longtable}{p{2.7cm}| p{11.1cm}}
\caption{Inputs and outputs parameters of
  mdap\_gandalf.pro} \label{dap_tab:mdap_gandalf}
\\ \hline \endfirsthead

\hline
\endhead

\hline
\endlastfoot

%\begin{tabular}{p{2.7cm}| p{2.5cm} |p{9cm}}
\hline {\bf INPUTS} & \\ \hline 

TEMPLATES & vector containing the spectrum of a single template star
         or array of dimensions nPixels\_Templates x nTemplates containing
         different templates to be optimized during the fit of the kinematics.
         nPixels\_Templates has to be $>=$ the number of pixels sampling the
         galaxy spectrum nPixels\_Galaxy.\\
%
GALAXY & nPixels\_Galaxy elements vector containing the spectrum of
        the galaxy to be measured.

        Both star and the galaxy spectra HAVE to be logarithmically
        rebinned on a natural ln-base (or in log10, see LOG10
        keyword). Warning: the MaNGA DAP uses a natural ln-base.\\
%
NOISE & nPixels\_Galaxy elements vector containing the 1*sigma error
        of the emission spectrum. If this is not available an array of
        constant unity values should be passed.\\
%
VELSCALE & Float. Velocity scale of the spectra in km/s per pixel. It has
       to be the same for both the galaxy and the template spectra.\\
%   
SOL & on INPUT it must be a vector containing the stellar ; kinematics
      needed to convolve the input stellar spectrum or ; template library
      (V, $\sigma$, h3, h4, h5, and h6). On OUTPUT it will contain the
      results of the gas fit with weights assigned to the multiplicative
      polynomial appended.\\

EMISSION\_SETUP. & A structure containing the index, the name, and
       the wavelength of the fitted emission lines. It must also
       contain the starting/input values for the line amplitudes,
       velocities, and widths of the lines, as well as keywords
       specifying whether each line is a part of a doublet, and
       whether its position and width fit freely, to be hold at its
       input values, or to be tied that of another line.\\
%   
L0\_GAL & Float. the ln-lambda value corresponding to the starting pixel in
       the galaxy spectrum.\\
%   
L0\_STEP. Float. the ln-lambda step of corresponding to the pixels
       sampling the galaxy spectrum. 
   
      Warning:  Wavelengths are assumed to be in Angstrom. \\
\hline
{\bf KEYWORDS}  & \\
\hline
DEGREE & Integer. degree of the Legendre polynomial used for correcting the
       template continuum shape during the fit (default: -1). Warning:
       the DAP uses DEGREE = -1.\\
%
MDEGREE & Integer. degree of the Legendre polynomial used for correcting
       the template continuum shape during the fit (default: 0).
       This correction is MULTIPLICATIVE. Warning: the DAP sets it automatically
       to 0 if the REDDENING is fitted.\\
%
GOODPIXELS &  integer vector containing the indices of the pixels in
       the galaxy spectrum (in increasing order) that will be
       included in the fit. IMPORTANT: in all likely situations this
       keyword *has* to be specified.\\
%
INT\_DISP & Nlines elements vector, where N\_lines is the number of
           emission li nes defined in the user-input emission line set-up file
           (see \ref{dap_sec:configure}). It contains the instrumental velocity
           dispersion for each line (in km/sec), as function of its observed
           wavelength (gas velocity starting guess is used to calculate the
           observed wavelength).\\
%
LOG10: allows to deal with data that have been logarithmically
       rebinned in lambda, using a base 10. Warning: the DAP uses 
       natural log base. \\
%
REDDENING & 1 or 2 elements array. It allows to include in the fit the
          effect of reddening by dust, by specifying a single $E(B-V)$ guess for
          extinction, or a two-element array of $E(B-V)$ guesses. A single guess
          will trigger the use of a single-screen dust model, affecting the
          entire spectrum, whereas passing a two-elements array of guesses will
          add a second dust component affecting only the nebular fluxes. This
          second option is recommended only when recombination lines are clearly
          detected, so that a temperature-dependent prior (e.g. the decrement of
          the Balmer lines) on their relative intensity can be use to constrain
          such a second, internal component.\\
%
L0\_TEMPL & the ln-lambda value corresponding to the starting pixel
       in the template spectra. Needed if using the REDDENING keyword
       and the DUST\_CALZETTI function. \\
%
FOR\_ERRORS & A keyword specifying whether we wish errors to be
       properly estimated. Warning: in the DAP this keyword is always activated.\\
%
   /QUIET: set this keyword to mute the output.\\
%
  /PLOT: set this keyword to plot the best fitting solution at the end
  of the fit. Warning: do not USE it in the DAP.\\
%
\hline
{\bf OUTPUT PARAMETERS}  & \\
\hline
SOL & A 4xNlines vector containing the values of the best fitting
       flux, amplitude, Velocity, and velocity dispersion for each
       emission line. If the keyword INT\_DISP has been set the velocity
       dispersion is already the intrinsic one, otherwise it will be the
       observed width of the line.\\
%
BESTFIT & a named variable to receive a vector containing sum of
       best fitting stellar (convolved by the input LOSVD) and
       emission-line Gaussian templates. This model include bending
       of the stellar templates by the best fitting multiplicative
       polynomials, and of additive polynomials, if specified.\\
%   
EMISSION\_TEMPLATES: a named variable to receive an
       [nPixels\_Galaxy, Nlines] array containing the best fitting emission-lines
       templates. The emission spectrum can be obtained from this
       array by simply doing total(emission\_templates,2).\\
%
WEIGHTS& a named variable to receive the value of the weights by which
       each template was multiplied to best fit the galaxy spectrum.\\
%
ERROR& a named variable that will contain a vector of formal
       errors (1 sigma) for the parameters in the output vector
       SOL.

       If the FOR\_ERRORS keyword IS NOT specified no errors for the
       line amplitudes and fluxes will be returned and the
       uncertainties on the position and width of the lines should be
       regarded only as order of magnitude estimates.  If the
       FOR\_ERRORS keyword IS specified (Default in the DAP), a second
       fully non-linear fit of all the emission-line parameter will
       provide correct estimates for the errors on all emission-line
       parameters.  Still, Keep in mind that these errors are
       meaningless unless Chi$^2$/DOF$\sim$1. If a constant noise
       spectrum is provided, the formal errors will be automatically
       rescaled under the assumption the model is a good
       representation of the data.\\
%
/fix\_gas\_kin &  If set, the gas kinematics are not fitted. The 
                   return value is that of the starting guesses.\\ 
%
range\_v\_gas & 2 elements array]. It specifies the boundaries for the gas best 
           fit velocity (in km/sec). Default: starting\_guess $\pm$ 2000 km/sec.\\
%
range\_s\_gas &  2 elements array]. It specifies the boundaries for the gas best fit 
             velocity dispersion (in km/sec). Default: $21 < \sigma < 499$ km/sec.\\
%
external\_library & String that specifies the path to the external FORTRAN library, 
                   which contains the fortran versions of mdap\_bvls.pro. If not 
                   specified, or if the path is invalid, the default internal IDL 
                   mdap\_bvls code is used. \\
\hline
\hline
\end{longtable}
\end{center}

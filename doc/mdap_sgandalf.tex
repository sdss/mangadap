\subsection{{\tt mdap\_sgandalf.pro}}
\label{dap_sec:mdap_sgandalf}

This procedure is called by mdap\_spectral\_fit.pro and it fits an
input galaxy spectrum with a set of stellar templates and Gaussian
emission lines to get the absorption and emission line kinematics, the
emission lines fluxes and equivalent widths, the weighs of the stellar
templates, the best fit models for stars and gas, the best fitting
stellar template and the best fitting stellar template convolved by
the LOSVD, the residuals, and the reddening (if required).

The kinematics are recovered parametrically, using a Gauss function
plus high order Gauss-Hermite moments for the stellar kinematics, and
a Gaussian function for the emission line kinematics.

The code itself is an implementation of the ppxf code by M. Capellari
(insert reference), and the spectroscopic decomposition code by
L. Coccato \citep{Coccato+11}.

The input galaxy spectrum is fitted according to the following steps:

\begin{enumerate}

  \item an optimal stellar template is build as linear combination of
    stars in a library. Regularization over the stellar population
    properties can be used.

  \item an optimal emission line spectrum is build as linear
    combination of N Gaussian emission lines. The number N of the
    emission lines and their rest-frame wavelength are read from an
    input file. This step should account for instrumenal LSF (to be
    developed).

  \item the optimal stellar template is convolved with a Gauss-Hermite
    line of sight velocity distribution, parametrized by V, $\sigma$,
    h3, and h4.

  \item the optimal emission line spectrum is convolved with a
    Gaussian line of sight velocity distribution, parametrized by V
    and $\sigma$.

  \item the convolved optimal emission line spectrum and optimal
    template are multiplied by set of legendre polynomials, or by a
    reddening function, parametrized by $E(B-V)$ according to the
    mdap\_dust\_calzetti.pro function (Section
    \ref{dap_sec:mdap_dust_calzetti}).

  \item spectra from points 4 and 5 are added together and compared to
    the input galaxy spectrum.

  \item points from 1 to 6 are repeated till minimum $\chi^2$ is
    reached.

\end{enumerate}

The list of input/output parameters is given in Table
\ref{dap_tab:mdap_sgandalf}.


\begin{center}
\begin{longtable}{p{2.7cm}| p{11.1cm}}
\caption{Inputs and outputs parameters of mdap\_sgandalf} \label{dap_tab:mdap_sgandalf} \\
\hline
\endfirsthead
\hline
\endhead
\hline
\endlastfoot
\hline
{\bf  INPUTS} & \\
TEMPLATES1 & [dbl array].  Vector containing the spectrum of a single template star or
        more ; commonly an array of dimensions
        TEMPLATES[nPixels,nTemplates] containing ; different templates
        to be optimized during the fit of the kinematics.  ; nPixels
        has to be $\geq$ the number of galaxy pixels.  

         Tips:
           \begin{itemize}

   \item To apply linear regularization to the WEIGHTS via the
     keyword REGUL, TEMPLATES should be an array of two
     TEMPLATES[nPixels, nAge], three TEMPLATES[nPixels, nAge, nMetal]
     or four TEMPLATES[nPixels,nAge,nMetal,nAlpha] dimensions,
     depending on the number of population variables one wants to
     study.  This can be useful to try to attach a physical meaning
     to the output WEIGHTS, in term of the galaxy star formation
     history and chmemical composition distribution.  In that case
     the templates may represent single stellar population SSP
     models and should be arranged in sequence of increasing age,
     metallicity or alpha along the second, third or fourth
     dimension of the array respectively.

   \item TEMPLATES and GALAXY do not need to span the same
     wavelength range. However an error will be returned by
     SGANDALF, if the velocity shift in pixels, required to match
     the galaxy with the templates, becomes larger than nPixels. In
     that case one has to truncate either the galaxy or the
     templates to make the two rest-frame spectral ranges more
     similar.

      \end{itemize} \\
%
TEMPLATES2: &[N x 2  dbl array]. It contains:

  \begin{itemize} 
    \item TEMPLATES2[*,0] the values of the wavelenghts of the N
      emission lines to fit, in logarithmic units (ln or Log10 as in
      the input galaxy spectrum).
    \item TEMPLATES2[*,1] The N signs of the gas lines to fit, +1 for
      emission lines, -1 for absorption lines (e.g. NaI.)
   \end{itemize} \\
%
   GALAXY: & [dbl array]. It contains the spectrum of the galaxy to be measured. The
       star and the galaxy spectra have to be logarithmically rebinned but the
       continuum does *not* have to be subtracted. The rebinning may be
       performed with the mdap\_do\_ log\_rebin.pro (Section \ref{dap_sec:mdap_do_log_rebin}).
 For high redshift galaxies, one should bring the spectra
       close to the restframe wavelength, before doing the SGANDALF
       fit, to prevent too large velocity shifts of the
       templates. This can be done by dividing the observed
       wavelenghts by (1+z), where z is a rough estimate of the
       galaxy redshift, before the logarithmic rebinning. TO BE
       IMPLEMENTED IN THE PIPELINE.\\
%
NOISE: & [dbl array vector]. It contains the 1*sigma error (per pixel)
     in the galaxy spectrum. IMPORTANT: the penalty term of the
     sgandalf method is based on the *relative* change of the fit
     residuals. For this reason the penalty will work as expected even if
     no reliable estimate of the NOISE is available (see Cappellari \&
     Emsellem [2004] for details).  If no reliable noise is available this
     keyword can just be set to: NOISE = galaxy*0+1 ; Same weight for all
     pixels. \\
% 
VELSCALE: & [float]. Velocity scale of the spectra in km/s per pixel. It
     has to be the same for both the galaxy and the template spectra.\\
%
  START: & [6 elements vector]. [velStart\_ stars,sigmaStart\_ stars, h3, h4, velStart\_ gas, sigmaStart\_ gas] 
      with the initial estimate for the velocity and the velocity dispersion in km/s.\\
%
\hline
{\bf OPTIONAL INPUTS} &   \\
%
BIAS: & [float]. This parameter biases the (h3,h4,...) measurements towards zero
       (Gaussian LOSVD) unless their inclusion significantly decreses the
       error in the fit. Set this to BIAS=0.0 not to bias the fit (Default value used in the DAP): the
       solution (including [V,$\sigma$]) will be noisier in that case. The
       default BIAS should provide acceptable results in most cases, but it
       would be safe to test it with Monte Carlo simulations. This keyword
       precisely corresponds to the parameter $\backslash$lambda in the Cappellari \&
       Emsellem (2004) paper. Note that the penalty depends on the *relative*
       change of the fit residuals, so it is insensitive to proper scaling
       of the NOISE vector. A nonzero BIAS can be safely used even without a
       reliable NOISE spectrum, or with equal weighting for all pixels.\\
%
DEGREE: & [integer]. degree of the *additive* Legendre polynomial used to correct
       the template continuum shape during the fit.
       Default: DEGREE = -1, i.e. no additive polynomial are fitted.\\
% 
GOODPIXELS:& [integer array]. It contains the indices of the good
       pixels in the GALAXY spectrum (in increasing order). Only these
       pixels are included in the fit. If the /CLEAN keyword is set,
       in output this vector will be updated to contain the indices of
       the pixels that were actually used in the fit.\\
%: 
LAMBDA: & [dbl array]. When the keyword REDDENING is used, the user has
      to pass in this keyword a vector with the same dimensions of GALAXY,
      giving the restframe wavelength in Angstrom of every pixel in the
      input galaxy spectrum, i.e. lambda = EXP(logLam).\\
%
MDEGREE: & [integer]. degree of the *multiplicative* Legendre polynomial (with mean of 1)
       used to correct the continuum shape during the fit (default: 0). The
       zero degree multiplicative polynomial is always included in the fit as
       it corresponds to the weights assigned to the templates.
       Note that the computation time is longer with multiplicative
       polynomials than with the same number of additive polynomials.
       IMPORTANT: Multiplicative polynomials cannot be used when
       the REDDENING keyword is set.\\
%
   MOMENTS: & [integer]. Order of the Gauss-Hermite moments to fit. Set this keyword to 4
       to fit [h3,h4] and to 6 to fit [h3,h4,h5,h6]. Note that in all cases
       the G-H moments are fitted (nonlinearly) *together* with [V,sigma].

       If MOMENTS=2 or MOMENTS is not set then only [V,sigma] are
       fitted and the other parameters are returned as zero.

       If MOMENTS=0 then only the templates and the continuum additive
       polynomials are fitted and the WEIGHTS are returned in output.\\
%
  REDDENING: & [Float]. Set this keyword to an initail estimate of the reddening $E(B-V)>=0$
      to fit a positive reddening together with the kinematics and the templates.
      After the fit the input estimate is replaced with the best fitting $E(B-V)$ value.
      The fit assumes the exctinction curve of Calzetti et al. (2000, ApJ, 533, 682)
      but any other prescriptions could be trivially implemented by modifying the
      function SGANDALF\_REDDENING\_CURVE within the procedure.

      IMPORTANT: The MDEGREE keyword cannot be used when REDDENING is set. \\
%
   REGUL:  & [Float]. If this keyword is nonzero, the program applies second-degree
       linear regularization to the WEIGHTS during the SGANDALF fit.
       Regularization is done in one, two or three dimensions depending on whether
       the array of TEMPLATES has two, three or four dimensions respectively.
       Large REGUL values correspond to smoother WEIGHTS output. The WEIGHTS tend
       to a linear trend for large REGUL. When this keyword is nonzero the solution
       will be a trade-off between smoothness of WEIGHTS and goodness of fit.

       The effect of the regularization scheme is to enforce the numerical second 
       derivatives between neighbouring weights (in every dimension) to be equal 
       to -w[j-1]+2*w[j]-w[j+1]=0 with an error Delta=1/REGUL. It may be helpful 
       to define REGUL=1/Delta and view Delta as the regularization error.

      IMPORTANT: Delta needs to be of the same order of magnitude as the typical 
       WEIGHTS to play an effect on the regularization. One way to achieve this is: 
     \begin{itemize}
       \item divide the TEMPLATES array by a scalar in such a way that the typical 
         template has a median of one (e.g. TEMPLATES/=median(TEMPLATES)); 
       \item do the same for the input GALAXY spectrum (e.g. GALAXY/=median(GALAXY)). 
         In this situation Delta and REGUL should be *roughly* of order unity. 
     \end{itemize}

     Here is a possible recipe for chosing the regularization parameter REGUL:
     \begin{itemize}
          \item Perform an un-regularized fit (REGUL=0) and then rescale the input
              NOISE spectrum so that $\chi^2$/DOF = $\chi^2$/N\_ELEMENTS(goodPixels) = 1.
              This is achieved by rescaling the input NOISE spectrum as
              NOISE = NOISE*sqrt($\chi^2$/DOF) = NOISE*sqrt(SOL[6]);
         \item Increase REGUL and iteratively redo the sgandalf fit until the $\chi^2$
              increases from the unregularized $\chi^2$ = N\_ELEMENTS(goodPixels)
              value by $\Delta chi^2$ = sqrt(2*n\_elements(goodPixels)).
     \end{itemize}
       The derived regularization corresponds to the maximum one still consistent
       with the observations and the derived star formation history will be the
       smoothest (minimum curvature) that is still consistent with the observations.

       For a detailed explanation see Section 18.5 of Press et al. (1992,
       Numerical Recipes 2nd ed.) available here http://www.nrbook.com/a/bookfpdf.php.
       The adopted implementation corresponds to their equation (18.5.10).\\
%
   VSYST: & [Float]. Difference in km/sec between the first pixel of the ln-wavelenght of the template stars and 
       the ln-wavelenght of the galaxy, i.e. dv = (loglam\_templates[0]-loglam\_gal[0])*c (it is computed in mdap\_spectral\_fitting.pro.\\ 
{\bf OPTIONAL KEYWORDS} &   \\
   /CLEAN &  set this keyword to use the iterative sigma clipping method
       described in Section 2.1 of Cappellari et al. (2002, ApJ, 578, 787).
       This is useful to remove from the fit unmasked bad pixels, residual
       gas emissions or cosmic rays. IMPORTANT: This is recommended *only* if a reliable estimate of the
       NOISE spectrum is available. See also note below for SOL.\\
%
   /QUIET &   set this keyword to suppress verbose output of the best fitting
       parameters at the end of the fit.\\
\hline
{\bf OUTPUTS} &  \\ 
  SOL: &[9+MDEGREE elements vector]. It contains  the values of
       [Vel\_star, Sigma\_Star, h3, h4, h5, h6,$\chi^2$/DOF, Vel\_gas, Sigma\_gas] of the best fitting solution, where DOF
       is the number of Degrees of Freedom (number of fitted spectral pixels).
       Vel is the velocity, Sigma is the velocity dispersion, h3-h6 are the
       Gauss-Hermite coefficients. The model parameter are fitted simultaneously.

       The following safety limits on the fitting parameters are hardcoded(check!!):
      \begin{enumerate}
         \item Vel is constrained to be +/-2000 km/s from the first input guess
         \item velScale/10 < Sigma < 1000 km/s
         \item -0.3 < [h3,h4,...] < 0.3 (limits are extreme value for real galaxies)
           \end{enumerate}

       IMPORTANT: if $\chi^2$/DOF is not $\sim$1 it means that the errors are not
       properly estimated, or that the template is bad and it is *not* safe
       to set the /CLEAN keyword.

      When MDEGREE > 1 then SOL contains in output the 9+MDEGREE elements
       [Vel\_star, Sigma\_Star, h3, h4, h5, h6,  $\chi^2$/DOF,Vel\_gas,Sigma\_gas, 
       cx1, cx2, ..., cxn], where cx1, cx2,...,c xn
       are the coefficients of the multiplicative Legendre polynomials
       of order 1, 2, ..., n. The polynomial can be explicitly evaluated as:

           x = range(-1d,1d,n\_elements(galaxy))

           mpoly = 1d ; Multiplicative polynomial

           for j=1,MDEGREE do mpoly += legendre(x,j)*sol[6+j]

      When the reddening correction is used, SOL contains 10
       elements [Vel\_star,Sigma\_Star, h3, h4, h5, h6, $\chi^2$/DOF, Vel\_gas,Sigma\_gas, ebv], where ebv 
       is the best fitting reddening value.\\
%;
gas\_intens &[dbl array].  It contains the intensities of the emission lines (corrected for reddening if the reddening is fitted).\\
%;
gas\_fluxes  &[dbl array].  It contains the fluxes of the emission lines (corrected for reddening if the reddening is fitted).\\
%;
gas\_ew &[dbl array].  It contains the equivalent widths of the emission lines (corrected for reddening if the reddening is fitted).\\
%;
gas\_intens\_ err &[dbl array]. It contains the errors associated to gas\_intens.\\
%;
gas\_fluxes\_ err &[dbl array]. It contains the errors associated to gas\_fluxes.\\
%;
gas\_ew\_err & [dbl array]. It contains the errors associated to gas\_ew.\\
%;
\hline
{\bf OPTIONAL OUTPUTS} &  \\ 
 BESTFIT: &[dbl array]. A named variable to receive a vector with the best fitting
       model: this is a linear combination of the templates,
       multiplied by multiplicative pols (if any) or reddening
       corrected (if required), convolved with the best fitting
       LOSVD, with added polynomial continuum terms
       and the best fitting gas emission lines.\\
%
 BF\_COMP1 &[dbl array]. A  named variable to receive a vector with the best fitting
       model for the stellar component: this is a linear combination
       of the stellar templates,
       multiplied by multiplicative pols (if any) or reddening
       corrected (if required), convolved with the best fitting
       LOSVD. It does not contain additive pols.\\
%
 BF\_COMP2  &[dbl array]. A  named variable to receive a vector with the best fitting
        ionized gas kinematics, convolved with the gas LOSVD, and
        mutiplied by the reddening curve (if applicable).\\
%
 OPT\_TEMPL &[dbl array]. A  named variable to reveive a vector containing the optimal
       template. This is the linear combination of the templates,
       multiplied by multiplicative pols (if any) or reddening corrected
       (if required), NOT CONVOLVED with the stellar LOSVD. It does
       not contain additive pols.\\
%
 MPOLY &[dbl array]. A  named variable to receive a vector with the multiplicative
       pol that has been multipliedto BF\_COMP1 and BESTFIT.\\
%
 ADDITIVE\_POL &[dbl array]. A  named variable to receive a vector with the additive
      pol that has been added to BESTFIT. \\
%
ERROR: &[dbl array]. A  named variable that will contain a vector of *formal* errors
       (1*sigma) for the fitted parameters in the output vector SOL.\\
%
POLYWEIGHTS: &[Flt array]. A named variable to receive the weights of the additive Legendre polynomials.
       The best fitting additive polynomial can be explicitly evaluated as

           x = range(-1d,1d,n\_elements(galaxy))

           apoly = 0d ; Additive polynomial

           for j=0,DEGREE do apoly += legendre(x,j)*polyWeights[j]

     When doing a two-sided fitting (see help for GALAXY parameter), the additive
       polynomials are allowed to be different for the left and right spectrum.
       In that case the output weights of the additive polynomials alternate between
       the first (left) spectrum and the second (right) spectrum.\\
%
   WEIGHTS  &[float array] a named variable to receive the value of the weights by which each stellar
       template and ionized gas template was multiplied to best fit
       the galaxy spectrum.
        Stellar weights are WEIGHTS[0:Ntemplates1-1]
        Gas emission lines intensities are  WEIGHTS[Ntemplates1 : *]
        N.B. Gas intensities are de-reddened (i.e. the intensity in
        the input spectrum is lower than WEIGHTS[Ntemplates1 : *],
        because it account for the reddening at that wavelength.
        If reddening is not fitted, then intensities and weights
        are the same.\\
%
\hline
\end{longtable}
\end{center}


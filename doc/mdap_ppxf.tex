\section{{\tt mdap\_ppxf.pro}}
\label{dap_sec:mdap_ppxf}

This module fits the input galaxy spectrum with a series of stellar
templates to get the best fit values of stellar kinematics, and
stellar weights.

The original code is by M. Cappellari and E. Emsellem (Cappellari \&
Emsellem, 2004). The following modifications to the original code have
been performed to adapt it to the MaNGA DAP workflow.

\begin{itemize}

\item Inclusion of the optional inputs range\_v\_star, and
  range\_s\_star to set the boundaries (with respect the stating
  guesses) where the best fit stellar velocity and velocity dispersion
  should be searched.

\item Inclusion of the keyword /fix\_star\_kin, to fix the stellar
  kinematics to the input values.

\item Inclusion of the optional keyword /external\_library, to run the
  fortran version of the BVLS module.

\end{itemize}


Table \ref{dap_tab:mdap_ppxf} lists the inputs and outputs of
mdap\_ppxf.pro (from the original ppxf.pro description, with the
addition of the new implemented keywords).


\begin{center}
\begin{longtable}{p{2.7cm}| p{11.1cm}}
\caption{Inputs and outputs parameters of
  mdap\_ppxf.pro} \label{dap_tab:mdap_ppxf}
\\ \hline \endfirsthead

\hline
\endhead

\hline
\endlastfoot

%\begin{tabular}{p{2.7cm}| p{2.5cm} |p{9cm}}
\hline
{\bf  INPUTS} &  \\
\hline
   TEMPLATES &  N elements vector containing the spectrum of a single template star or more
       commonly an array of dimensions TEMPLATES[nPixels,nTemplates] containing
       different templates to be optimized during the fit of the kinematics.
       nPixels has to be >= the number of galaxy pixels.

      \smallskip 

     - To apply linear regularization to the WEIGHTS via the keyword REGUL,
       TEMPLATES should be an array of two {\tt TEMPLATES[nPixels, nAge]}, three
       TEMPLATES[nPixels,nAge,nMetal] or four {\tt TEMPLATES[nPixels, nAge, nMetal, nAlpha]}
       dimensions, depending on the number of population variables one wants to study.
       This can be useful to try to attach a physical meaning to the output WEIGHTS, in
       term of the galaxy star formation history and chmemical composition distribution.
       In that case the templates may represent single stellar population SSP models
       and should be arranged in sequence of increasing age, metallicity or alpha along 
       the second, third or fourth dimension of the array respectively.

      \smallskip 

     - TEMPLATES and GALAXY do not need to span the same wavelength range. However
       an error will be returned by PPXF, if the velocity shift in pixels,
       required to match the galaxy with the templates, becomes larger than
       nPixels. In that case one has to truncate either the galaxy or the
       templates to make the two rest-frame spectral ranges more similar. \\
%
   GALAXY &  N elements vector containing the spectrum of the galaxy to be measured. The
       star and the galaxy spectra have to be logarithmically rebinned but the
       continuum does {\bf not} have to be subtracted. 

      \smallskip 

     - For high redshift galaxies, one should bring the spectra close
       to the restframe wavelength, before doing the PPXF fit, to
       prevent too large velocity shifts of the templates. This can be
       done by dividing the observed wavelenghts by (1+z), where z is a
       rough estimate of the galaxy redshift, before the logarithmic
       rebinning (Warning: not yet implemented in the DAP).

      \smallskip 

     - GALAXY can also be an array of dimensions {\tt GALAXY[nGalPixels,2]} containing
       two spectra to be fitted, at the same time, with a reflection-symmetric
       LOSVD. This is useful for spectra taken at point-symmetric spatial
       positions with respect to the center of an equilibrium stellar system.
       For a discussion of the usefulness of this two-sided fitting
       see e.g. Section 3.6 of Rix \& White (1992, MNRAS, 254, 389).
       (Warning: this feature has not been tested in the context of the DAP workflow)

      \smallskip 

     - IMPORTANT: 1) For the two-sided fitting the VSYST keyword has to be used.
       2) Make sure the spectra are rescaled to be not too many order of
       magnitude different from unity, to avoid over or underflow problems
       in the calculation. E.g. units of erg/(s cm$^2$ \AA) may cause problems!\\
%
   NOISE &  N elements vector containing the 1*sigma error (per pixel) in the galaxy spectrum.
       If GALAXY is a Nx2 array, NOISE has to be an array with the same dimensions.
 
      \smallskip 

    - IMPORTANT: the penalty term of the pPXF method is based on the *relative*
       change of the fit residuals. For this reason the penalty will work as
       expected even if no reliable estimate of the NOISE is available
       (see Cappellari \& Emsellem [2004] for details).
       If no reliable noise is available this keyword can just be set to:
           {\tt NOISE = galaxy*0+1} ; Same weight for all pixels.\\
%
   VELSCALE & Float. velocity scale of the spectra in km/s per pixel. It has to be the
       same for both the galaxy and the template spectra.\\
%
   START &  two elements vector {\tt [velStart, sigmaStart]} with the initial estimate
       for the velocity and the velocity dispersion in km/s.

      \smallskip 

      - Unless a good initial guess is available, it is recommended to set the starting
       {\tt sigma >= 3*velScale} in km/s (i.e. 3 pixels). In fact when the LOSVD is severely
       undersampled, and far from the true solution, the chi$^2$ of the fit becomes weakly
       sensitive to small variations in sigma (see pPXF paper). In some instances the
       near-constancy of chi$^2$ may cause premature convergence of the optimization.

      \smallskip 

      - In the case of two-sided fitting a good starting value for the
       velocity is velStart=0.0 (in this case VSYST will generally be nonzero).
       Alternatively on should keep in mind that velStart refers to the first
       input galaxy spectrum, while the second will have velocity -velStart.\\
%
\hline
{\bf  KEYWORDS} &  \\
\hline
   BESTFIT& a named variable to receive a vector with the best fitting
       template: this is a linear combination of the templates, convolved with
       the best fitting LOSVD, with added polynomial continuum terms.\\
%
   BIAS & This parameter biases the (h3,h4,...) measurements towards zero
       (Gaussian LOSVD) unless their inclusion significantly decreses the
       error in the fit. Set this to {\tt BIAS=0.0} not to bias the fit: the
       solution (including [V,sigma]) will be noisier in that case. The
       default BIAS should provide acceptable results in most cases, but it
       would be safe to test it with Monte Carlo simulations. This keyword
       precisely corresponds to the parameter $\lambda$ in the Cappellari \&
       Emsellem (2004) paper. Note that the penalty depends on the *relative*
       change of the fit residuals, so it is insensitive to proper scaling
       of the NOISE vector. A nonzero BIAS can be safely used even without a
       reliable NOISE spectrum, or with equal weighting for all pixels.\\
%
   /CLEAN & set this keyword to use the iterative sigma clipping method
       described in Section 2.1 of Cappellari et al. (2002, ApJ, 578, 787).
       This is useful to remove from the fit unmasked bad pixels, residual
       gas emissions or cosmic rays.

       \smallskip

     - IMPORTANT: This is recommended *only* if a reliable estimate of the
       NOISE spectrum is available. See also note below for SOL.\\
%
   DEGREE & degree of the *additive* Legendre polynomial used to correct
       the template continuum shape during the fit (default: 4).
       Set DEGREE = -1 not to include any additive polynomial.\\
%
   ERROR & a named variable that will contain a vector of *formal* errors
       (1*sigma) for the fitted parameters in the output vector SOL. This 
       option can be used when speed is essential, to obtain an order of 
       magnitude estimate of the uncertainties, but we *strongly* recommend to 
       run Monte Carlo simulations to obtain more reliable errors. In fact these 
       errors can be severely underestimated in the region where the penalty 
       effect is most important (sigma $<$ 2*velScale).

       \smallskip

     - These errors are meaningless unless Chi$^2$/DOF$\sim$1 (see
     parameter SOL below).  However if one *assume* that the fit is
     good, a corrected estimate of the errors is: errorCorr =
     error*sqrt(chi$^2$/DOF) = error*sqrt(sol[6]).

       \smallskip

     - IMPORTANT: when running Monte Carlo simulations to determine the error,
       the penalty (BIAS) should be set to zero, or better to a very small value.
       See Section 3.4 of Cappellari \& Emsellem (2004) for an explanation.\\
%
   GOODPIXELS & integer vector containing the indices of the good pixels in the
       GALAXY spectrum (in increasing order). Only these pixels are included in
       the fit. If the /CLEAN keyword is set, in output this vector will be updated
       to contain the indices of the pixels that were actually used in the fit.

      \smallskip

     - IMPORTANT: in all likely situations this keyword *has* to be specified.\\
%
   LAMBDA & When the keyword REDDENING is used, the user has to pass in this
       keyword a vector with the same dimensions of GALAXY, giving the restframe
       wavelength in Angstrom of every pixel in the input galaxy spectrum.\\
%
  MDEGREE & degree of the *multiplicative* Legendre polynomial (with mean of 1)
       used to correct the continuum shape during the fit (default: 0). The
       zero degree multiplicative polynomial is always included in the fit as
       it corresponds to the weights assigned to the templates.
       Note that the computation time is longer with multiplicative
       polynomials than with the same number of additive polynomials.
     - IMPORTANT: Multiplicative polynomials cannot be used when
       the REDDENING keyword is set.\\
%
   MOMENTS &  Order of the Gauss-Hermite moments to fit. Set this keyword to 4
       to fit [h3, h4] and to 6 to fit [h3,h4,h5,h6]. Note that in all cases
       the G-H moments are fitted (nonlinearly) *together* with [V, sigma].
 
      \smallskip

    - If MOMENTS=2 or MOMENTS is not set then only [V,sigma] are
       fitted and the other parameters are returned as zero.

      \smallskip

     - If MOMENTS=0 then only the templates and the continuum additive
       polynomials are fitted and the WEIGHTS are returned in output.

      \smallskip

       Warning: The DAP workflow fits only moments up to h4. \\
%
   /OVERSAMPLE & Set this keyword to oversample the template by a factor 30x
       before convolving it with a well sampled LOSVD. This can be useful to
       extract proper velocities, even when sigma < 0.7*velScale and the 
       dispersion information becomes totally unreliable due to undersampling. 
       IMPORTANT: One should sample the spectrum more finely is possible, 
       before resorting to the use of this keyword! \\

      \smallskip
 Warning: not tested in the DAP workflow.
%
   /PLOT &  set this keyword to plot the best fitting solution and the residuals
       at the end of the fit. Warning: DO NOT USE IN THE DAP WORKFLOW.\\
%
   POLYWEIGHTS & vector with the weights of the additive Legendre polynomials.
       The best fitting additive polynomial can be explicitly evaluated as
  
    \medskip

        \ \ \    {\tt x = range(-1d,1d,n\_elements(galaxy))}

        \ \ \    {\tt apoly = 0d} ; Additive polynomial

        \ \ \    {\tt for j=0,DEGREE do apoly += legendre(x,j)*polyWeights[j]}

        \medskip

     - When doing a two-sided fitting (see help for GALAXY parameter), the additive
       polynomials are allowed to be different for the left and right spectrum.
       In that case the output weights of the additive polynomials alternate between
       the first (left) spectrum and the second (right) spectrum.\\
%
   /QUIET & set this keyword to suppress verbose output of the best fitting
       parameters at the end of the fit. \\
%
   REDDENING & Set this keyword to an initail estimate of the reddening $E(B-V)>=0$
       to fit a positive reddening together with the kinematics and the templates.
       After the fit the input estimate is replaced with the best fitting $E(B-V$) value.
 
     - IMPORTANT: The MDEGREE keyword cannot be used when REDDENING is set.

       Warning: This keyword is NEVER used in the DAP workflow: the reddening is measure usiong Gandalf. \\
%

   REGUL & (Warning: not tested in the DAP workflow) If this keyword is nonzero, the program applies second-degree
       linear regularization to the WEIGHTS during the PPXF fit.
       Regularization is done in one, two or three dimensions depending on whether
       the array of TEMPLATES has two, three or four dimensions respectively.
       Large REGUL values correspond to smoother WEIGHTS output. The WEIGHTS tend
       to a linear trend for large REGUL. When this keyword is nonzero the solution
       will be a trade-off between smoothness of WEIGHTS and goodness of fit.

      \smallskip

     - The effect of the regularization scheme is to enforce the
     numerical second derivatives between neighbouring weights (in
     every dimension) to be equal to {\tt -w[j-1]+2*w[j]-w[j+1] = 0}
     with an error {\tt Delta=1/REGUL}. It may be helpful to define
     REGUL=1/Delta and view Delta as the regularization error.

      \smallskip

      - IMPORTANT: Delta needs to be of the same order of magnitude as the typical 
       WEIGHTS to play an effect on the regularization. One way to achieve this is: 
       (i) divide the TEMPLATES array by a scalar in such a way that the typical 
       template has a median of one (e.g. TEMPLATES/=median(TEMPLATES)); 
       (ii) do the same for the input GALAXY spectrum (e.g. GALAXY/=median(GALAXY)). 
       In this situation Delta and REGUL should be *roughly* of order unity.  
 
      \smallskip

     - Here is a possible recipe for chosing the regularization parameter REGUL:

     \begin{enumerate}
         \item Perform an un-regularized fit (REGUL=0) and then rescale the input
              NOISE spectrum so that Chi$^2$/DOF = Chi$^2$/N\_ELEMENTS(goodPixels) = 1.
              This is achieved by rescaling the input NOISE spectrum as
              {\tt NOISE = NOISE*sqrt(Chi$^2$/DOF) = NOISE*sqrt(SOL[6])};
         \item Increase REGUL and iteratively redo the pPXF fit until the Chi$^2$
              increases from the unregularized Chi$^2$ = N\_ELEMENTS(goodPixels)
              value by $\Delta$Chi$^2$ = sqrt(2*n\_elements(goodPixels)).
     \end{enumerate}

      The derived regularization corresponds to the maximum one still consistent
       with the observations and the derived star formation history will be the
       smoothest (minimum curvature) that is still consistent with the observations.

      \smallskip

       - For a detailed explanation see Section 18.5 of Press et al. (1992,
       Numerical Recipes 2nd ed.) available here http://www.nrbook.com/a/bookfpdf.php.
       The adopted implementation corresponds to their equation (18.5.10).\\
% 
 SKY & Warning: not tested in the DAP. vector containing the spectrum
       of the sky to be included in the fit, or array of dimensions
       SKY[nPixels,nSky] containing different sky spectra to add to
       the model of the observed GALAXY spectrum. The SKY has to be
       log-rebinned as the GALAXY spectrum and needs to have the same
       number of pixels.

      \smallskip

     - The sky is generally subtracted from the data before the PPXF fit. However,
       for oservations very heavily dominated by the sky spectrum, where a very
       accurate sky subtraction is critical, it may be useful *not* to subtract
       the sky from the spectrum, but to include it in the fit using this keyword.\\
%
   VSYST  & difference between the log-lam values of the wavelength of the galaxy and the template stars, expressed in km/sec.\\
%
   WEIGHTS & a named variable to receive the value of the weights by which each
       template was multiplied to best fit the galaxy spectrum. The optimal
       template can be computed with an array-vector multiplication:
    \smallskip

           T\ \ \    {\tt EMP = TEMPLATES \# WEIGHTS} (in IDL syntax)
    \smallskip

       - When the SKY keyword is used WEIGHTS[0:nTemplates-1] contains the weights
       for the templates, while WEIGHTS[nTemplates:*] gives the ones for the sky.
       In that case the best fitting galaxy template and sky are given by:
    \smallskip

          \ \ \    {\tt TEMP = TEMPLATES \# WEIGHTS[0:nTemplates-1]}

          \ \ \    {\tt BESTSKY = SKY \# WEIGHTS[nTemplates:*]}
    \smallskip

       - When doing a two-sided fitting (see help for GALAXY parameter) *together*
       with the SKY keyword, the sky weights are allowed to be different for the
       left and right spectrum. In that case the output sky weights alternate
       between the first (left) spectrum and the second (right) spectrum.\\
%
/fix\_star\_kin &        If set, the stellar kinematics are not
                       fitted. The return value is that of the starting guesses.\\ 
%
range\_v\_star & 2 elements array]. It specifies the boundaries for the stellar best 
           fit velocity (in km/sec). Default: starting\_guess $\pm$ 2000 km/sec.\\
%
range\_s\_star &  2 elements array]. It specifies the boundaries for the stellar best fit 
     velocity dispersion (in km/sec). Default: $21 < \sigma < 499$ km/sec.\\
%
external\_library & String that specifies the path to the external FORTRAN library, which 
                  contains the fortran versions of mdap\_bvls.pro. 
                  If not specified, or if the path is invalid, the default internal IDL mdap\_bvls code is used. \\
\hline
{\bf  OUTPUTS} &  \\
\hline
  SOL & seven elements vector containing in output the values of {\tt
     [Vel, Sigma, h3, h4, h5, h6, Chi$^2$/DOF]} of the best fitting
     solution, where DOF is the number of Degrees of Freedom (number of
     fitted spectral pixels).  Vel is the velocity, Sigma is the
     velocity dispersion, h3-h6 are the Gauss-Hermite coefficients. The
     model parameter are fitted simultaneously.  Warning: in the DAP
     workflow, only moments up to h4 are fitted.

    \smallskip

     - I hardcoded the following safety limits on the fitting parameters:
     \begin{itemize}
         \item Vel is constrained to be $\pm$2000 km/s from the first input guess
         \item  $velScale/10 < Sigma < 500$ km/s
         \item  $-0.4 < [h3 ,h4, ...] < 0.4$ (limits are extreme value for real galaxies)
         \end{itemize}
    \smallskip

     - IMPORTANT: if Chi$^2$/DOF is not ~1 it means that the errors are not
       properly estimated, or that the template is bad and it is *not* safe
       to set the /CLEAN keyword.

    \smallskip

     - When MDEGREE > 1 then SOL contains in output the 7+MDEGREE
     elements {\tt [Vel, Sigma, h3, h4, h5, h6, Chi$^2$/DOF, cx1, cx2,
         ..., cxn]}, where cx1, cx2, ..., cxn are the coefficients of
     the multiplicative Legendre polynomials of order 1, 2, ...,
     n. The polynomial can be explicitly evaluated as:
       
       \medskip
   
          \ \ \  \ {\tt x = range(-1d,1d,n\_elements(galaxy))}
      
          \ \ \  \ {\tt mpoly = 1d }; Multiplicative polynomial
      
          \ \ \  \ {\tt for j=1,MDEGREE do mpoly += legendre(x,j)*sol[6+j]}\\
\hline
\hline
\end{longtable}
\end{center}

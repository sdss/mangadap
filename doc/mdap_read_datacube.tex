\subsection{{\tt mdap\_read\_datacube.pro}}
\label{dap_sec:mdap_read_datacube}

This interface module reads the input datacube or RSS files (which
must be multi-layer fits file) and extract from it all the arrays and
quantities, which are needed for the analysis. Table
\ref{dap_tab:mdap_read_datacube} list the inputs and outputs required
for this module.



\begin{center}
\begin{longtable}{p{2.7cm}| p{11.1cm}}
\caption{Inputs and outputs parameters of mdap\_read\_datacube.pro} \label{dap_tab:mdap_read_datacube} \\
\hline
{\bf  INPUTS} & \\
\hline
\endfirsthead
\hline
\endhead
\hline
\endlastfoot
\hline
datacube\_name & string with the name of the fits file containing the input in datacube or RSS formats. 
                         For inputs in datacube format, the file must be a fits file with the following
                         extensions:
                                \begin{enumerate}
                             \item flux in 1e-17 ergs/s/cm$^{-2}$/\AA. This extension must be a 3D
                               array, with the wavelength direction along the 3rd axis. 
                            \item  Inverse variance associated to the first extension
                            \item  Wavelength solution (1D dbl array). Constant linear step
                              (preferred)? Constant Log-step? Constant ln-step?
                            \item  RSS format (row-stacked spectra) (NOT USED)                  
                            \item  coverage map; (NOT USED)
                           \end{enumerate}

                         For inputs in RSS format, the file must be a fits file with the following
                         extensions:
                          \begin{enumerate}
                            \item flux in 1e-17 ergs/s/cm$^{-2}$/\AA. This extension must be a 3D
                               array, with the wavelength direction along the 3rd axis.
 
                            \item Inverse variance associated to the first extension

                            \item Wavelength solution (1D dbl array). Constant linear step
                              (preferred)? Constant Log-step? Constant ln-step?
                            
                            \item X position table.  Since position is a function of wavelength this is an 
                               array of the same size as the flux array.  Coordinates should be in arcseconds 
                               relative to the center of the data cube (i.e., roughly the center of the galaxy).
                            \item Y position table.
                            \end{enumerate}\\
%
\hline
{\bf  OPTIONAL INPUTS}  & \\
\hline
lrange & [2 elem vector]. It indicates the wavelentgh range (in angstrom) where to
       extract the information for Signal and Noise. Default: use the entire spectra range \\
%
\hline {\bf OPTIONAL KEYWORDS} &  \\ 
\hline 
/keep\_original\_ step & If set, the wavelength output vector will be the same as the one
                         define from the input fits file. The default
                         is to re-construct (and  therefore re-inrpolate the galaxy and error
                         spectra) the output wavelength vector with constant ang/pixel step
                         using the minimum ang/pixel step that is stored in the wavelength
                         solution. For MANGA data, it is suggested to set this keyword on.\\
%
 /use\_total           &   If set, the signal is the total of the counts in the selected wavelength range, the noise is the sum in
                         quadrature of the noises in the selected range. Useful for emission lines.\\
\hline
{\bf  OUTPUTS} &  \\
\hline
data           &  Galaxy spectra. [NxMxT elements array] in the case that the inputs are in datacube format, or [NxT elements array]
               in the case the inputs are in RSS format. Spectra are resampled over the vector wavelength.    \\
%
error          & Errors associated to data. [NxMxT elements array] in the case that the inputs are in datacube format, or [NxT elements array]
               in the case the inputs are in RSS format. Spectra are resampled over the vector wavelength. \\
%
wavelength     & [T elements array]. Wavalenght value (in angstrom)  where data and errors are computed. The vector is constructed with constant 
               linear step in ang/pixel 
               (unless the /keep\_original\_ step keyword is selected). If input spectra have a logarithmic sampling, 
               the minimum available step is used (e.g. log\_lam[1]-log\_lam[0], converted into angstrom).  In the case of MANGA data, 
               we recommend to set the /keep\_original\_ step keyword, to have the wavelength output vector equal to the wavelength vector of the data.\\
%
x2d            & Array containing the x coordinates in arcseconds (0 is the center of the field of view). 
               [NxM elements array] in the case that the inputs are in datacube format, or 
               [N elements array] in the case the inputs are in RSS format. \\
%
y2d            & Array containing the y coordinates in arcseconds (0 is the center of the field of view). 
               [NxM elements array] in the case that the inputs are in datacube format, or 
               [N elements array] in the case the inputs are in RSS format. \\
%
signal         & Mean galaxy signal per \AA, obtained considering all the wavelength range 
               (or only the range specified by lrange), for each input spectrum.               
               [NxM elements array] in the case that the inputs are in datacube format, or 
               [N elements array] in the case the inputs are in RSS format. 
               The signal is computed as the median of each spectrum, in the wavelength range specified by lrange (unless the keyword /use\_total is set). \\
\\
%
noise          & Mean galaxy error per \AA, obtained considering all the wavelength range 
               (or only the range specified by lrange)), for each input spectrum. Calculation is done on original spectra, not 
               those resampled over the vector wavelenght.
               [NxM elements array] in the case that the inputs are in datacube format, or 
               [N elements array] in the case the inputs are in RSS format.
               The signal is computed as the median of each error, in the wavelength range specified by lrange (unless the keyword /use\_total is set).\\
%               
cdelt1         & [double].        Spatial sampling along x direction (arcsec/pixel). If inputs are in RSS format, it is set to 0.5 arcsec/pixel.  \\
%               
cdelt2         & [double].        Spatial sampling along y direction (arcsec/pixel). If inputs are in RSS format, it is set to 0.5 arcsec/pixel.  \\
%               
header2d       & [str array].     The header for output two-dimensional maps produced by the DAP. \\
\hline
{\bf  OPTIONAL OUTPUTS} &  \\
\hline
%
 x2d\_ reconstructed  &   [NxM] elements array]   if input is in DATACUBE format,  and x2d\_ reconstructed = x2d.

                          [N'xM'] elements array] if inputs are in RSS format, the x2d coordinates are resampled over a 2D map with fixed 0"5/pixel sampling
                                     and define the  x2d\_ reconstructed map.\\
%
 y2d\_ reconstructed &    [NxM] elements array   if input is in DATACUBE format,  and y2d\_reconstructed = y2d.

                          [N'xM'] elements array if inputs are in RSS format, the y2d coordinates are resampled over a 2D map with fixed 0"5/pixel sampling
                                     and define the  y2d\_ reconstructed map.\\
%
 signal2d\_ reconstructed  &  [NxM] elements array if input is in DATACUBE format, and  signal2d\_reconstructed = signal.

                         [N'xM'] elements array if inputs are in RSS format, the signal is resampled over the 2D map defined by
                                         x2d\_ reconstructed  and y2d\_ reconstructed. \\
%
 noise2d\_ reconstructed  &   [NxM] elements array if input is in DATACUBE format, and noise2d\_reconstructed = noise.

                        [N'xM' array] If inputs are in RSS format, the noise is resampled over the 2D map defined by
                                         x2d\_reconstructed  and y2d\_reconstructed. \\
%
 version   & [string]            Module version. If requested, the module is not execute and only version flag is returned.\\
%
\hline
\end{longtable}
\end{center}

Note: NaN values in each error vector will be replaced by the median
error value (computed using error's defined values).


\subsubsection{Future developments}

The following items needs to be implemented:

\begin{itemize}
  \item Implement with new format of input datacube (i.e. read and handlemask arrays).
  \item Allow for identification and removal of foreground stars.
  \item Get Milky Way extinction, and provide extinction-corrected galaxy spectra and errors. 
        Insert an appropriate main module (i.e. dust\_getval.pro) to do this
  
\end{itemize}

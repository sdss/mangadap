\section{{\tt mdap\_kinemetry.pro}}
\label{dap_sec:mdap_kinemetry}

This main module implements the kinemetry.pro module by Kajinovic et
al. 2008) to measure the rotation curve (stars and gas), position angle, and axial ratio and amplitude of
inflows/outflows.

Table \ref{dap_tab:mdap_kinemetry} lists the inputs and outputs of
mdap\_kinemetry.pro (from the original module by D. Krajinovic).


\begin{center}
\begin{longtable}{p{2.7cm}| p{11.1cm}}
\caption{Inputs and outputs parameters of
  mdap\_kinemetry.pro} \label{dap_tab:mdap_kinemetry}
\\ \hline \endfirsthead

\hline
\endhead

\hline
\endlastfoot

%\begin{tabular}{p{2.7cm}| p{2.5cm} |p{9cm}}
\hline
{\bf  INPUTS} &  \\
\hline
%
XBIN   & 1D array with X coordinates describing the map.\\
%
YBIN  & 1D array with Y coordinates describing the map.\\
%
MOMENT & 1D array with kin.moment (e.g. velocity) values  at XBIN, YBIN positions.\\
%
\hline
{\bf  OPTIONAL INPUTS KEYWORDS  } &  \\
\hline
NTRM  & scalar specifying the number of terms for the harmonic analysis 
	           of the profile extracted along the best fitting ellipse. Default
	           value is 6 odd terms, which means the following terms will be 
	           used in the expansion: a1, b1, a3, b3, a5 and a5, or the terms
                   corresponding to sin(x), cos(x), sin(3x), cos(3x), sin(5x), cos(5x).
                    Warning: the DAP uses NTRM=2.\\
%
ERROR & 1D array with errors to VELBIN values. If this
             keyword is specified then the program will calculate
	           formal (1sigma) errors of the coefficients which are
	           returned in ER\_PA, ER\_Q (determined by MPFIT), ER\_CF 
	           (determined by SVD) variables. 

              If IMG keyword is set ERROR has to be a 2D array. If if
              is not supplied, it is creatred as 2D arrays with all
              values equal to 1. \\  
%
SCALE  & scalar specifying the pixel scale on the map. If not set, the
         SAURON pixel scale (0.8 arcsec) is assumed. Warning: DAP uses scale=1.\\
%
IMG    & 2D array containing an image. This keyword was designed specifically
	           for surface photometry (analysis of the zeroth moment of the LOSVD), 
	           to increase the spead of calculations, but if kinematic map is large 
	           (and has many pixels) it can be also passed through this keyword and 
	           analysis will be quicker. To use kinemetry for photometry it is also 
	           necessary to set keyword EVEN and it can be useful to use NTRM=10
 	         in order to measure disky/boxy deviations (4th terms). 
	           Images are very different from current kinematic maps. They are much larger
            and usually not binned, so make regular 2D grids. This is the reason to
	           treat the kinematic maps and images in a different way. If IMG is set,
	           the treatment follows the ideas of Jedrzejewsky (1987): inner parts of
	           (small radii, rad < 40 pixels) are interpolated while outer parts are 
            binned in elliptical sectors (64 in angle to increase the signal and of 
            thicknes equal to 10\% of the given radius). When IMG is used center can 
            be also fitted (currently only in photometry, so if keyword EVEN is not 
            set - which usually means one is analysing a velocity map - center is not fitted). 
            An estimate of the center is given through XC and YC keywords. ERROR should be a 2D array
            of the same size as IMG. When IMG is used, VELCIRC and VELKIN keywords contain 
            reconstructed images. It is assumed that image coordiantes are in pixels 
            (not physical units). Keyword SCALE is automatically set to 1. If this is not 
            the case, set SCALE to a required number. In order to be compatible with 
            previous versions of kinemetry XBIN, YBIN and MOMENT still have to be passed
            but they can be dummmy 1D variables, unless certain image areas are masked 
            (as bad pixels). IF keyword BADPIX is used, XBIN and YBIN should be 1D arrays 
            with the actual coordinates of the IMG. One can use the following set of lines 
            to make the arrays: 

            \smallskip

             \ \ \        {\tt s=size(img)}  

             \ \ \        {\tt n=s[1]*s[2]}

             \ \ \        {\tt yy=REPLICATE(1,s[1])\#(indgen(s[2]))}

             \ \ \        {\tt xx=(indgen(s[1]))\#REPLICATE(1,s[2])}

             \ \ \        {\tt xbin=REFORM(xx, n)}

             \ \ \        {\tt ybin=REFORM(yy, n)} \\
X0  & an estimate of the X coordinate of the center (in pixels). If not given X0=0.
	           For accurate determination of the center and other ellipse parameters at small
	           radii it is important the ellipse includes the center of the galaxy. \\

Y0  & an estimate of the Y coordinate of the center (in pixels). If not given Y0=0.\\
%
FIXCEN & keyword, if set center will be fixed and no attempt will be made during the
            harmonic decomposition to find new center. Center is fixed to X0 and Y0. 
            This keyword is optional only for photometry (or even moments in general). 
            For ODD moments, center is fixed always. \\
%
NRAD  & scalar specifying the number of radii along which kinemetry should be run.
            IF not specified, NRAD=100. Kinemetry will stop when the edge of the map
	           is encountered and NRAD is not necessary achived. To force kinemetry to 
	           do all radii, relax condition in keyword COVER.\\
%
NAME & name of the object (used by VERBOSE keyword and for internal
            plotting).\\
%
PAQ  & 2 element or 2*NRAD element vector specifying position angle (PA)
	           and flattening (q) of the ellipses in that order (kept constant).
	           It is possible to specify a set of PA and q values (that 
	           correspond to given radii (see RADIUS keyword)), for which one
	           wants to get the Fourier coefficients. In this case PAQ should
	           be set as follows: PAQ=[PA1, Q1, PA2, Q2...., PAnrad, Qnrad]

            It can be also used as an initial condition for determination of ellipses. 
            In this case, it should be called together with /NOGRID keyword (currently  
            implemented only for photometry). 
            IF PAQ keyword is used to define ellipses along which harmonic
            decomposition is made, then keyword NOGRID should not be used. In this case 
            center is fixed (and should be defined via X0 and Y0 keywords if IMG keyword
            is used).\\
%
NOGRID & keyword, if set it bypasses the direct minimisation via a grid in PA
	           and Q values. It should be used together with PAQ parameters, when
	           a good estimate of PA and Q are passed to the program, but not if PAQ
            serve to pre-define ellipses for harmonic decomposition.
	           It is desigend with photometry in mind, where the problem usually has
           only one well defined minimum (in PA,Q plane). It speeds up the
	           calculation, but for the higher kinematic moments it is not as robust and 
            it is advised not to be used (first use the grid to find the best fit 
            PA and Q values.).\\
%
NPA   & scalar specifying the number of PA used to crudely estimate
            the parameters of the best fit ellipse before entering
            MPFIT. Default value is 21. To speed up the process and for
	           quick tests it is useful to use a small number (e.g 5). 
            Complicated maps may require a bigger number (e.g. 41).\\
%
NQ  & scalar specifying the number of q used to crudely estimate
            the parameters of the best fit ellipse before entering
            MPFIT. Default value is 21. To speed up the process and for
	           quick tests it is useful to use a small number (e.g 5). 
            Complicated maps may require a bigger number (e.g. 41).\\
%   
RANGEQ & 2 element vector specifying the min and max value for 
	           flattening Q. Default values are 0.2 and 1.0.\\
%   
RANGEPA & 2 element vector specifying the min and max value for 
	           position angle PA. Default values are -90 and 90 (degrees).\\
%
BADPIX & 1D array containing indices of pixels which should not be
            used during harmonic fits. This keyword is used when data 
            are passed via IMG. It is usefull for masking stars and
            bad pixels. When used, XBIN and YBIN should be real 
            coordinates of the pixels of IMG (see IMG for more details). 
            The bad pixels are passed to the routine which defines/selects
            the ellipse coordiantes (and values) to be fitted, and they are 
            removed from the subsequent fits. All pixels of
            the ellipse which are 2*da from the bad pixels are removed 
            from the array, where da is the width of the ring. \\ 
%
/ALL & If this keyword is set then the harmonic analysis of the rings 
	           will include both even and odd terms. If this keyword is set,
	           and NTRM = n then the following terms are used in the expansion:
	           a1, b2, a2, b2, a3, b3,...., an, bn (or coeffs nex to: sin(x),
	           cos(x), sin(2x), cos(2x), sin(3x), cos(3x),...,sin(nx),cos(nx)).\\
%
/EVEN & set this keyword to do kinemetry on even kinematic moments. 
	           In this case, kinemetry reduces to photometry and the best
	           fitting ellipse is obtained by minimising a1, b1, a2, b2
	           terms. When this keyword is set, keyword /ALL is automatically
	           set and NTRM should be increased (e.g. NTRM=10 will use the 
	           following terms in the expansion: a1, b2, a2, b2, a3, b3, a4, b4
	           (or coeffs. next to sin(x),cos(x),sin(2x), cos(2x),sin(3x),
	           cos(3x),sin(4x),cos(4x))).\\
%
/VSYS & if this keyword is set the zeroth term (a0) is not extracted.
	           (for odd moments).This might be useful for determinatio of
	           rotation curves.One can first run kinemetry without setting 
	           this keyword to find the systemic velocity (given as cf[*,0]). 
                   Then subtract the systemic velocity form the velocity map and 
	           re-run kinemetry with /vsys set. In this case the zeroth terms 
	           will be zero. For completeness, it is also possible to input 
	           VSYS, e.g. VSYS=10. The zeroth term will not be calculated, 
	           but it will be set to 10 in output. Given that Fourier terms
	           are orthogonal, it should not be necessary to set this keyword
	           in general.\\ 
%
RING & scalar specifying desired radius of the first ring. Set this
	           keyword to a value at which the extraction should begin. This 
	           is useful in case of ring-like structures sometimes observed in 
	           HI data.\\
%   
RADIUS & 1D array with values specifying the lenght of the semi-major axis
	           at which the data (kin.profile) should be extracted from the map 
	           for the kinemetric analisys. The values should be in pixel space 
            (not in physical units such as arcsec).  If this keyword is set, 
	           the values are coopied into the output variable: RAD.\\
%   
COVER & Keyword controling the radius at which extraction of
	           values from the map stops. Default value is 0.2, meaning
	           that if less than 20\% of the points along an ellipse are
	           are not present, the program stops.\\ 
%
/BMODEL & If this keyword is set, a model moment map is constructed. This keyword
            should be set together with VELCIRC and VELKIN keywords, which will
	           contain the reconstructed map, using the first dominant term and all 
	           terms, respectively. IF IMG keyword is used, the outputs are 2D images,
 	         otherwise BMODEL reconstructs the map at each input position XBIN,YBIN.
	           If BMODEL is not set VELCIRC and VELKIN will contain reconstructed values
	           at the positions of XELLIP and YELLIP.\\
%
/PLOT  & If this keyword is set, diagnostic plots are shown for each radii: 

		       - the best ellipse (overploted on kin.moment map). If IMG
		         keyword is set, the image is scaled to the size of the ellipse.
		         The centering of the overplotted ellipse is good to 0.5 pixels
		         so for small radii (r < a few pixels) it is possible that the 
		         position of the center of the overplotted ellipse is not on the
		         brightes pixel. 

		       - PA-Q grid with the position of the minimum
                        (green diamond) for the parameters of the best
                        fit ellipse determined by MPFIT, where the
                        initial (input to MPFIT) values of PA and Q are
                        presented by the grid of dots and the colours
                        show the Chi2 square contours (linearly
                        interpolated between the PA,Q points),

		       - fit to kin.profile (white are the DATA, red is the FIT, where
		         FIT is given by a0+b1*cos(x) for odd, and a0 for even moments), 

		       - residuals (DATA - FIT), and overplotted higher order
		         terms (green: a1,a3 and b3, red: a1,a3,b3,a5 and b5; 
		         for the /EVEN case - green: a1, b1, a2, b2, red: a1, b1, a2, b2, a4, b4)

                         \smallskip

                        Warning: do not use in the DAP.\\
%
/VERBOSE & set this keyword to print status of the fit on screen
            including information on:
\begin{itemize}
               \item Radius - number of the ring that was analysed
               \item RAD    - radius of the analysed ring (if SCALE is passed 
                          it is given in the same units, otherwise in pixels)
               \item PA     - position angle of the best fitting ellipse
               \item Q      - flattening of the best fitting ellipse
               \item Xcen   - X coordinate of the centre of the ellipse
               \item Ycen   - Y coordinate of the centre of the ellipse
               \item Number of ellipse elements - number of points to which the data
                          points in the ring are sampled before derivation of
                          the best fit parameters and harmonic analysis. It 
                          varies between 20 and 64 (or 100 in non IMG model)
                          depending on the ring size, giving a typical sampling 
                          of 5.6 (3.6) degrees.
\end{itemize}\\
\hline
{\bf OUTPUT}  & \\
\hline
   RAD & 1D array with radii at which kin.profiles were extracted.\\
%
   PA  & 1D array with position angle of the best fitting ellipses,
            PA is first determined on an interval PA=[-90,90], where
            PA=0 along positive x-axis. Above x-axis PA $>$ 0 and below
            x-axis Pa $<$ 0. PA does not differentiate between receding
            and approaching sides of (velocity) maps. This is
            transformed to the usual East of North system, where the 
	           East is the negative x-axis, and the North is the positive 
	           y-axis. For odd kin.moments PA is measured from the North 
	           to the receding (positive) side of the galaxy (which is 
	           detected by checking the sign of the cos(theta) term. For
	           the even terms it is left degenerate to 180 degrees rotation.\\
%
Q  & 1D array with flattening of the best fitting ellipses
            (q=1-ellipticity), defined on interval q=[0.2,1]\\
%
CF  & 2D array containing coefficients of the Fourier expansion
            for each radii cf=[Nradii, Ncoeff]. For example: 
	           a0=cf[*,0], a1=cf[*,1], b1=cf[*,2]....\\
\hline
{bf OPTIONAL OUTPUT KEYWORDS} & \\
\hline
VELKIN & 1D array of reconstructed kin.moment using NTRM harmonic
            terms at positions XBIN,YBIN, obtained by linear interpolation
	           from points given in XELLIP and YELLIP keywords (if BMODEL keyword
	           is used, otherwise at positions XELLIP and YELLIP.\\
%   
VELCIRC & 1D array containinng 'circular velocity' or a0 + b1*cos(theta)
            at positions XBIN, YBIN (velcirc = a0, in case of EVEN moments), 
	           obtained by linear interpolation from points given in XELLIP 
	           and YELLIP keywords (if BMODEL keyword is used, otherwise at 
	           positions XELLIP and YELLIP.\\
%
GASCIRC  &1D array containing circular velocity or Vcirc=cf[*,2]*cos(theta)
            at positions XBIN and YBIN, obtained for fixed PA and q. 
            PA and q are taken to be median values of the radial variation of PA and q. 
            IF keyword PAQ is used than GASCIRC give the circular velocity (no systemic
            velocity) for the median values of PA, Q values. Note that this is different
            from VELCIRC (also VELKIN) which is obtained on the best
            fitting ellipses and also includes Vsys (cf[*,0]) term. 
            This keyowrd is useful for gas velocity maps, if one wants to obtain 
            a quick disk model based on the circular velocity. \\
%
ER\_PA & 1D array of 1sigma errors to the ellipse position angle.\\
%
ER\_Q  &  1D array of 1sigma errors to the ellipse axial ratio.\\
%
ER\_CF  & 2D array containing 1 sigma errors to the coefficients 
	           of the Fourier expansion for each radii.\\

XELLIP & 1D array with X coordintes of the best fitting ellipses.\\
%
YELLIP & 1D array with Y coordintes of the best fitting ellipses.\\
%
XC    & the X coordinate of the center (in pixels). If X0 not fit XC=0.\\
%
YC    & the Y coordinate of the center (in pixels). If Y0 not fit YC=0.\\
\hline
\hline
\end{longtable}
\end{center}

